
\subsection{Joel von Rotz}
<< Durch diese Arbeit konnten wir neue Erfahrungen sammeln, welche uns im Berufsleben hilfreich sein können. Wir konnten schlussendlich durch eine umfangreiche Planung eine Wirbelstrombremse entwickeln, welche zwar nicht unseren Erwartungen entsprach, aber dafür neue Kenntnisse, insbesondere über Wirbelstrom und Magnetismus, brachte.
\newpara
Ich erlernte durch diese Arbeit das Dokumentationprogramm \textit{LATEX}. Mit diesem Programm können professionelle Arbeiten geschrieben werden und ich konnte diese Arbeit damit schreiben. Zusätzlich konnte ich einen Magneten anhand mathematischen Formeln berechnen und neue Erkenntnisse im Projektmanagement erlernen. 
\newpara
Mit den neuen Erkenntnissen bin ich bereit, dieses Projekt weiterzuführen. Ich werde bestehende Funktionen verbessern und neue Funktionen in das Projekt implementieren. Ich glaube daran, dass diese Arbeit mir auch in Zukunft neue Erfahrungen bringen wird, welche in weitere Projekte eingesetzt werden können.
\newpara
Trotz dieses Erfahrungserfolg, gerieten wir in Zeitdruck. Wir mussten zusammen innert kürzester Zeit eine Dokumentation  schreiben. Meine Lehre daraus ist, dass ich in Zukunft eine umfangreichere Planung erstellen werde. Reflektiv war dieses Projekt meiner Meinung nach eine gute Lehre, aus Fehlern zu lernen und auch neue Methoden aufzustellen, welche diese Fehler beheben können.
\newpara 
Meinen Erwartungen nach wollte ich die Arbeit vollständig abschliessen, was aber nicht möglich war. Ich bin daher über den Fortschritt nicht ganz zufrieden. Wir konnten einen Meilenstein setzen und ich will diesen weiterführen. 
>>
\subsection{Stefan Ruckli}
<< Unsere Arbeit wandelte sich von einer Grundidee zu einem Projekt, in dem jeder von uns gefragt war. Wir konstruierten schlussendlich eine funktionsfähige Wirbelstrombremse. Hiermit können wir von uns selbst behaupten, dass wenn ein elektrisch Leitfähiges Medium bei Bewegung, welches unter Einfluss eines äusseren Magnetfeld steht, darauf eine Kraft gegen die Bewegung bildet.
\newpara
Ich erwartete von unserem Projekt, dass wir durch Messungen effektiv einen Nachweis über diesen Physikalischen Effekt darlegen können. Dies war bei unserem Projekt durchaus der Fall, was mich faszinierte. Somit setzte ich mich auch gerne mit dem Konzept einer Wirbelstrombremse auseinander. Ich habe viele neue Erfahrungen im Bereich der Elektrotechnik. Darüber hinaus habe ich noch neue Kenntnisse in dem Bereich der Werkstück-Bearbeitung gesammelt.
\newpage
Für mich war die Arbeit, trotz unseres nicht komplett erfüllten Leitsatzes, ein Erfolg. Reflektiv kann ich behaupten, dass unser Projekt jede Erfahrung wert war. >>
\subsection{Filip Estermann}
<< Am Anfang unserer IDPA kannten wir uns noch nicht. Das Einzige, was wir voneinander wussten, war das gemeinsame Interesse an der Elektrizität. Aus diesem Interesse entstanden einige Ideen für unsere Arbeit und schliesslich entwickelte sich daraus ein Projekt. Wir waren alle motiviert, eine gute Arbeit zu machen und wir setzten uns hohe Ziele. Diese waren am Anfang etwas zu gross und wir reduzierten sie für unsere IDPA auf zwei Ziele, die zusammen eine überschaubare Arbeit werden sollen.
\newpara
Meine persönlichen Erwartungen an diese Arbeit waren grösser, als nur unsere Ziele zu erreichen. Ich wollte neue Erfahrungen im Bereich der Elektrotechnik sammeln und mich vor allem bei der Dokumentation weiterentwickeln, da ich bis jetzt noch nie eine solche geschrieben hatte. 
\newpara
Rückblickend kann ich sagen, dass unsere Arbeit ein voller Erfolg war. Wir konnten gut zusammenarbeiten und wir fanden auf auftretende Probleme meistens schnell eine Lösung. Auch wenn wir unser zweites Ziel nicht erreichten, konnten wir daraus trotzdem einiges lernen und es ist eine wertvolle Erfahrung. Meine Erwartungen konnte ich vollständig erfüllen. >>