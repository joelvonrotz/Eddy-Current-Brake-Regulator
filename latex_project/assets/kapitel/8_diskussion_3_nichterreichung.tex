\subsection{Nichterreichung}
Die Erreichung der Hypothese konnte aus folgenden Gründen nicht erreicht werden:
\begin{itemize}
  \item Fehlberechnungen
  \item Kenntnis über Regler
  \item Zeitdruck
\end{itemize}
\subsubsection{Fehlberechnungen}
Bereits im Kapitel \ref{cap:methoden_magnet_design} erwähnt, wurden die Formeln falsch angewendet. Dies hat den Grund, dass mit dem \textit{Trail-\&-Error}-Prinzip der Magnet dimensioniert wurde. Das Trägheitsdrehmoment und das Drehmoment von W.R. Smythes Formel \ref{equ:smythe_rotation_included} wurden nur miteinander verglichen, anstatt zusammenzuführen. Dies hat den Nachteil, dass der Magnet zu ungenau dimensioniert wird. Folgend wird die korrekte Zusammenstellung der Formel aufgeführt.
\newpara
Wird anhand den aufgelisteten Formeln im Kapitel \ref{cap:methoden_magnet_design} eine einzige Formel erstellt, würde die Berechnung des Magneten stark vereinfacht. Die Formel (\ref{equ:smythe_rotation_included}) wird mit den Formeln \ref{equ:magnetischer_fluss}, \ref{equ:magnetische_durchflutung}, \ref{equ:magnetischer_widerstand}, \ref{equ:magnetischer_serie_widerstand} und \ref{equ:tragheitsmoment_kombiniert} erweitert, damit kann eine einzige  Formel erstellt werden. Die daraus folgende Formel beinhaltet alle benötigten Parameter, welche für die Dimensionierung verwendet wird (Siehe Formel ).

\begin{equation}
  \label{equ:smythe_tragheit_kombiniert}
  \frac{m\cdot r^{2}\cdot \Delta\omega}{\Delta t}=\frac{n_{MAX}\cdot b\cdot \gamma\cdot (\frac{N\cdot I}{R_M})^{2}\cdot c^{2}}{a^{2}} \cdot (1-\frac{A^{2}\cdot a^{2}}{(A^{2}-c^{2})^{2}}) \tag{29}
\end{equation}
\newpara
Wird nach $R_M$ aufgelöst, erhält man (Formel \ref{equ:smythe_tragheit_RM_aufgelost})

\begin{equation}
  \label{equ:smythe_tragheit_RM_aufgelost}
  R_{M}=\frac{c\cdot I \cdot N}{a \cdot r}\cdot \frac{\sqrt{\Delta t \cdot n_{MAX}\cdot b \cdot \gamma \cdot (1-\frac{A^{2}\cdot a^{2}}{(A^{2}-c^{2})^{2}})}}{\sqrt{m\cdot\Delta\omega}} \tag{30}
\end{equation}
\newpara

$R_{M}$ beinhaltet Informationen über die Grösse des Magneten, der Aluminiumscheibe und des Luftspalts. Formel \ref{equ:magnetischer_serie_widerstand} wird mit der Formel des magnetischen Widerstands \ref{equ:magnetischer_widerstand} für den Luftspalt, für  die Aluminiumscheibe und für den Magneten erweitert (Siehe Formel \ref{equ:magnetischer_widerstand_eingesetzt}).

\begin{equation}
  \label{equ:magnetischer_widerstand_eingesetzt}
R_{M}=\frac{l_{Luft}}{\mu_{0}\cdot A}+\frac{l_{Alu}}{\mu_{0}\cdot \mu_{rAlu}\cdot A}+\frac{l_{Magnet}}{\mu_{0}\cdot \mu_{rMagnet}\cdot A} \tag{31}
\end{equation}
\newpara

Wird die Formel nach $l_{Magnet}$ aufgelöst, erhält man:

\begin{equation}
  l_{Magnet}=\frac{(R_{M}\cdot\mu_{0}\cdot\mu_{rAlu}\cdot A - l_{Luft}\cdot\mu_{rAlu}-l_{Alu})\cdot\mu_{rMagnet}}{\mu_{rAlu}} \tag{32}
\end{equation}
\newpara
Diese Formel ermöglicht nun, durch eine gemeinsame Querschnittsfläche, die Dicke der Scheibe und die Dicke des Luftspalts, die Magnetkreislänge zu berechnen. Anhand Magnetkreislänge und der Querschnittsfläche kann der Magnet dimensioniert werden.
\newpage
\subsubsection{Kenntnis über Regler}
Das zweite Problem geschah beim Austesten des Reglers. Beim Austesten wurden die berechneten Werte des Reglers aufgefangen und am Laptop angezeigt. Nur war nicht bekannt, was funktionsweise richtig war. Die im Anhang hinterlegten Tabelle mit dem Name \textit{results\_regler\_messung} zeigt Regelwerte im Bereich von 700 bis 800 an, was als zu hoch gilt. Diese Werte wurden anhand einem kleinen Test ermittelt, worin das Ziel war, den PID-Regler der Elektronik auszuprobieren (Siehe Abbildung \ref{fig:pid_regler_plot}).

\begin{figure}[ht]
    \begin{center}
      \begin{tikzpicture}
        \begin{axis}[
          width             = 12cm,
          compat            = 1.3,
          axis y line*      = left,
          ylabel            = {Drehzahl},
          xlabel            = {Zeit},
          legend pos        = north east,
          legend cell align = {left},
          x unit            = {\si{\second}},
          y unit            = {\si{1\per\second}},
          legend style      = {draw=none},
          xmax              = 40
        ]
        
          \addplot[line_red] table[x= time,y=electronic,col sep=comma,mark=none]{assets/diagramm/diskussion/pid_regler.csv}; 
          \label{plot:ist_drehzahl}           
          \addlegendentry{Ist-Drehzahl}
        
          \addplot[line_blue] table[x= time,y=solldrehzahl,col sep=comma,mark=none]{assets/diagramm/diskussion/pid_regler.csv}; 
          \label{plot:soll_drehzahl}
          \addlegendentry{Soll-Drehzahl}
        \end{axis}
        \begin{axis}[
          compat            = 1.3,
          width             = 12cm,
          axis y line*      = right,
          axis x line       = none,
          ylabel            = {Regelwert},
          ymin              = 680,
          ymax              = 760,
          xmax              = 100,
          legend pos        = north east,
          legend cell align = {left},
          legend style      = {draw = none},
          xmax              = 40
        ]
          \addlegendimage{/pgfplots/refstyle=plot:ist_drehzahl,color=red!70,line width= 1pt}
          \addlegendentry{Ist-Drehzahl}
          \addlegendimage{/pgfplots/refstyle=plot:soll_drehzahl,color=blue!70,line width= 1pt}
          \addlegendentry{Soll-Drehzahl}
          \addplot[line_black] table[x= time,y=pid_value,col sep=comma,mark=none]{assets/diagramm/diskussion/pid_regler.csv};
          \label{plot:regelwert_out}     
          \addlegendentry{Regelwert}
        \end{axis}
      \end{tikzpicture}
    \end{center}
    \vspace{-3ex}
    
    \caption{PID-Regelwerte mit Sollwert}
            
    \label{fig:pid_regler_plot}
  \end{figure}

Es kann in der Abbildung \ref{fig:pid_regler_plot} erkannt werden, dass der Regelwert abnimmt, obwohl die Drehzahl zunimmt (bei einer Zunahme müsste der Regelwert zunehmen). 
\newpara
Vermutlich sind die Grössen der Faktoren falsch gesetzt. Diese Werte konnten wegen zu wenig Zeit nicht geändert werden. Es gibt Rechenmethoden, welche für die Dimensionierung der Faktoren verwendet werden könnten, aber schlussendlich wird es mit dem Trial-\&-Error Prinzip enden. Es wird solange ausprobiert, bis geeignete Werte ermittelt werden können.
\newpara
Eine andere Vermutung ist, dass der Regler falsch implementiert wurde oder der verwendete Mikrocontroller die Werte nicht korrekt berechnet. 
\newpara
Der PID-Regler könnte in Zukunft anhand einer berechenbaren Situation überprüft werden. Dies wäre der Plan gewesen für diese Arbeit, aber da die Zeit nicht mehr reichte, wurde dies für die Zukunft geplant.

\subsubsection{Zeitdruck}
Der Hauptgrund, dass die Hypothese nicht erreicht werden konnte, war der Zeitdruck.
\newpara
Der Zeitplan, welcher für die Arbeit erstellt wurde, hat jeden einzelnen Schritt aufgelistet, welcher für die Arbeit ausgeführt werden muss, ausser die Radanpassung, welche während der der Arbeit dazwischen kam. Lieferzeiten und Fehlerbehebungen wurden ebenfalls nicht einberechnet, obwohl diese Schritte eigentlich wichtig sind.
\newpara
In anderen Worten, es kam zum Zeitdruck wegen einer Fehlplanung. Man hätte jeden Schritt detaillierter beschrieben sollen, worin der Ausführende direkt weiss, was gemacht werden muss. Dazu müsste die Planung für die Arbeit länger dauern und bis ins Detail geplant werden. Da dies bei der Arbeit nicht immer der Fall war, wurde zu viel Zeit verloren gegangen und als Folge kam es zum Stress.
\newpara
In Zukunft sollte genügend Zeit für die Planung investiert werden. Jeder Schritt müsste eine Beschreibung, Ziel und geschätzte Verarbeitungszeit besitzen, dass in der gesamten Zeitplanung die Schritte richtig verteilt werden können.
