\section{Abstract}\label{cap:abstract}
Das Ziel dieses Projektes war, eine Wirbelstrombremse zu bauen, die die Geschwindigkeit eines Rades bis zu fünf Prozent genau regeln kann. Dafür wurde ein Spinning Wheel (ein Fitnesstrainer) umgebaut und mit einem Elektromagneten, als Erzeuger der Bremskraft, erweitert. Dies wurde mit einer elektrisch leitenden Scheibe, die die Bremskraft auf das Rad übertragen soll, und mit einer Messeinrichtung, die die Frequenz, mit der sich das Rad dreht, bestimmt. Ausserdem wurde ein elektronischer Regler, der die Spannungsversorgung des Elektromagneten und dessen Bremskraft regeln soll, erstellt.
\newpara
Nach der Planung und nach den ersten Tests wurde analysiert, wo noch Probleme liegen und was angepasst werden muss: dass das Rad noch zu schwer war und verkleinert werden musste.
\newpara
Nach den entsprechenden Änderungen und Umbauten konnte die Wirbelstrombremse wunschgemäss eingesetzt werden und sie bremste das Rad relativ schnell ab. 
\newpara
Bei der Regelung wurde allerdings festgestellt, dass die Berechnung des Elektromagneten fehlerhaft war. Deshalb konnte die Regelung nicht in Betrieb genommen werden.
\newpara
In den folgenden Seiten erfahren Sie eine Einlesung in die wichtigsten Themen und die Gedanken der Verfasser dieser Arbeit.