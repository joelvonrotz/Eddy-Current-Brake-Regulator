\section{Einleitung}\label{cap:einleitung}
Eine Wirbelstrombremse nutzt das Magnetfeld des Bremsmagneten aus, um die Bremsscheibe abzubremsen. Da sie nur mit der induktiven (magnetischen) Bremskraft arbeitet, ist sie verschleissfrei, also benötigt keine Bremsklötze.
\newpara
 Die Ziele dieser Arbeit waren, eine gut funktionierende Wirbelstrombremse zu bauen und damit auch die Geschwindigkeit zu regeln. Dazu wurde folgende Hypothese formuliert: Eine Wirbelstrombremse, mit der die Geschwindigkeit auf 5\% Toleranz geregelt wird.
 \newpara
Dafür sollte eine elektrisch leitende Platte am Rad eines Spinning Wheels befestigt werden. Mit möglichst wenig Abstand zur der Platte, damit der magnetische Widerstand so gering wie möglich ist, soll ein Elektromagnet in Hufeisenform befestigt werden. Dieser Elektromagnet soll über einen Spannungsregler mit Spannung versorgt werden. Der Elektromagnet induziert in der elektrisch leitenden Platte eine Spannung, welche die Entstehung von Wirbelströmen und die Verzögerung der Platte zur Folge haben soll. Eine Messeinrichtung, bestehend aus einem Magneten, welcher am Rad befestigt ist, sowie einen Magnetschalter, welcher das vorbeibewegende Magnet erkennt, soll die Umdrehungsfrequenz des Rades bestimmen.
\newpara
Nach der Planung und der Materialbeschaffung wurden die Komponenten wie oben beschrieben zusammengebaut und die ersten Tests wurden vorbereitet. Um sicher zu gehen, dass die gemessenen Ergebnisse stimmen, wurde für die Tests eine zusätzliche Messeinrichtung installiert. Für diese Messeinrichtung wurde ein weisses Band um das Rad geklebt und eine Stelle wurde schwarz markiert. Ein Drehzahlmessgerät soll den Kontrast zwischen der schwarz markierten Stelle und dem weissen Band erkennen. Somit erkennt das Gerät, wann sich das Rad einmal im Kreis gedreht hat. Aus diesen Informationen berechnet es die Frequenz des Rades.
\newpara
Bei den ersten Tests funktionierte die Bremse nicht wunschgemäss. Die Bremsung war nur gering festzustellen. Nach der Fehleranalyse stand fest: Das Rad ist zu schwer und muss verkleinert werden, sodass es an Masse verliert.
\newpara
Nach dieser Anpassung wurde das Spinning Wheel erneut für die Tests aufgebaut. Bei diesen erneuten Tests stellten beide Messeinrichtungen einen deutlichen Unterschied zwischen der Verzögerung des Rades mit aktivierter und deaktivierter Wirbelstrombremse fest.
\newpara
Die Wirbelstrombremse funktionierte jetzt wunschgemäss und der Fokus lag nun auf dem zweiten Ziel, der Regelung. Dafür wurde ein elektronischer Regler erstellt. Dieser Regler vergleicht die am Rad gemessene Frequenz mit einer am Regler einstellbaren Frequenz. Anhand von diesen beiden Werten ermittelt der Regler, ob und wie stark das Rad abgebremst werden muss. Der Regler soll dann die Spannungsversorgung des Elektromagneten bestimmen: Muss das Rad stärker abgebremst werden, wird eine höhere Spannung an den Magneten angelegt. Durch die höhere Spannung soll der Elektromagnet eine höhere Bremskraft auf das Rad haben.
\newpara
Der Regler wurde in der Messeinrichtung und in der Spannungsversorgung eingebaut und das Spinning Wheel wurde erneut für die Tests aufgebaut. Die Tests ergaben, dass die Regelung nicht wunschgemäss funktionierte. Der Fehler lag bei der Dimensionierung des Elektromagneten. Dieser lieferte auf die von Konstantstromquelle gespiesene Spannung keine zufriedenstellenden Resultate.