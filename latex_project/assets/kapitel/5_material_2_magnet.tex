\subsection{Magnet}
Wichtiger Bestandteil eines Magneten ist das Metall. Das Metall muss ferromagnetisch sein, damit es etwas anziehen kann. Die Magnetisierbarkeit ist abhängig vom Eiseninhalt des Metalls. Je mehr Eisen das Metall hat, desto stärker wird das Magnetfeld bei gleichem Volumen sein (\cite{schulmaterial_magnetismus}). Also war das Ziel der Metall-Wahl, ein Metall herauszusuchen, welches sich für das magnetische Bremsen eignet.
\newpara

\newpara
Für diese Arbeit wurde das Metall \textbf{ETG-100} verwendet. Die Wahl wurde mit Hilfe einer externen Person ausgewählt, weil diese Person sich für das Thema interessierte und bereits selbst Wirbelstrombremsen entworfen hatte.
\newpara
Die Permeabilitätszahl, welche für den Magneten benötigt wird, konnte nur anhand einer Annahme gesetzt werden. Anhand eines Physikbuches wurde der Bereich für ferromagnetische Stoffe 300 bis 10'000 vorgenommen (\cite[S.661]{kuchling2014taschenbuch}). Als Annahme wurde daraus die Permeabilitätszahl 5'000 gesetzt.
