\subsection{Mechanik}
Die Hauptbedingung an das mechanische Gestell war, dass es ein drehbares Rad aufweisen soll, welches leicht in Bewegung gebracht werden kann. Ausserdem muss sich das Rad auch über eine längere Zeit selbst drehen können. Diese Bedingungen führten dazu, ein Spinning Wheel zu beziehen. Solch ein Spinning Wheel wird als Fitnessgerät verwendet. 
\newpara
Bei der Bremsscheibe wurde eine Aluminium-Scheibe verbaut. Dieses Metall wurde verwendet, da dies einen guten elektrischen Leitwert aufweist und somit auch gute Bremseigenschaften mit sich bringt. Ausserdem besitzt Aluminium als Leichtmetall ein relativ geringes spezifisches Gewicht.
\newpara
Hinter der Aluminium Scheibe befindet sich eine zweite, dickere Scheibe aus Automatenstahl. Hier wurde ETG 100 Automatenstahl verbaut, weil sich dieser leicht magnetisieren lässt und das Magetfeld vergleichsweise gut leiten kann.
\newpara
Als Haltegestell für den Elektromagneten wurden zwei parallel geführte C-Schienen verwendet, damit der Abstand zwischen dem Elektromagneten und der Bremsscheibe variiert werden kann. Ein ausserdem positiver Aspekt der C-Schienen ist, dass diese aus Aluminium bestehen und somit den Elektromagneten, ferromagnetisch von dem Eisengestell trennen. Somit kann eine höhere Wirkungskraft aus dem Magneten erzielt werden. Damit der Elektromagnet an seiner rechten Stelle bleibt, wurde dieser mit Schnellverlegern auf die Schiene geschraubt. Diese Schnellverleger ermöglichen ein rasches bewegen – und fixieren von dem Magnet. 
\newpara
Damit diese zwei C-Schienen sich in einer stabilen und korrekten Lage befinden, wurde ein Gestell gefertigt und die Schienen montiert. Dieses wurde aus dicken Blechplatten konstruiert. Aufgrund einer starken mechanischen Festigkeit wurde jenes Material verbaut. Dieses Eisengestell wurde mit Nieten an das Spinning Wheel gefügt.