\subsection{Physik}\label{cap:einlesung_physik}
\subsubsection{Drehmoment}
Bei einer Bremsung wird kinetische Energie in eine andere Energieform, meistens Wärmeenergie, umgewandelt. Durch diese Energieumwandlung sinkt die Geschwindigkeit des zu bremsenden Objektes (\cite{drehmoment}).
\newpara
Eine Geschwindigkeitsänderung im Verhältnis zu der dafür benötigten Zeit wird als Beschleunigung bezeichnet. 
\newpara
Eine Kraft setzt sich aus der Beschleunigung multipliziert mit der beschleunigten Masse zusammen. Daraus folgt: Je grösser die eingesetzte Kraft ist, desto grösser ist die Beschleunigung.
Da diese Kraft auf einen drehbaren, starren Körper ausgeübt wird, wird ein Drehmoment erzeugt. Ein Drehmoment ist das Produkt einer Kraft und dem senkrechten Abstand zum Drehpunkt. Die Wirkung eines Drehmoments entspricht der eines Kräftepaars. Das heisst, je grösser das Drehmoment ist, desto grösser ist seine Kraftwirkung.
\newpara
Aus diesen Erkenntnissen ergibt sich, dass die Position und die Kraft der Bremse eine wichtige Rolle spielen. Um aufgrund des Drehmoments eine optimale negative Beschleunigung zu erzeugen, sollte die Bremse deshalb möglichst weit von der Drehachse entfernt sein und die Kraft, welche im Falle der Wirbelstrombremse von Magneten erzeugt wird, sollte möglichst gross sein. Es spielen aber auch noch andere Faktoren wie die Induktion eine Rolle, wo die Bremse optimalerweise positioniert wird. Deshalb darf sie nicht ganz am Rand der zu bremsenden Scheibe sein (\cite{kuchling2014taschenbuch}).