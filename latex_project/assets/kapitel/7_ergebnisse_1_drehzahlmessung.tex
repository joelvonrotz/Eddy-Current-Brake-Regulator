\subsection{Drehzahlmessung} \label{cap:ergebnisse_drehzahlmessung}
\begin{figure}[ht]
    \begin{center}
      \begin{tikzpicture}
        \begin{axis}[
          width             = 10cm,
          compat            = 1.3,
          axis y line*      = left,
          ylabel            = {Drehzahl},
          xlabel            = {Zeit},
          legend pos        = south east,
          legend cell align = {left},
          x unit            = {\si{\second}},
          y unit            = {\si{1\per\second}},
          legend style      = {draw=none}
        ]
        
          \addplot[line_red] table[x= time,y=electronic,col sep=comma,mark=none]{assets/messurement_results/tempo.csv}; 
          \label{plot:tempo_electronic}           
          \addlegendentry{Drehzahl Elektronik}
        
          \addplot[line_blue] table[x= time,y=mess_converted,col sep=comma,mark=none]{assets/messurement_results/tempo.csv}; 
          \label{plot:tempo_measured}
          \addlegendentry{Drehzahl Messgerät}
        \end{axis}
        \begin{axis}[
          compat            = 1.3,
          width             = 10cm,
          axis y line*      = right,
          axis x line       = none,
          ylabel            = {Verhältnis},
          ymin              = 0,
          ymax              = 2,
          legend pos        = south east,
          legend cell align = {left},
          legend style      = {draw = none}
        ]
          \addlegendimage{/pgfplots/refstyle=plot:tempo_electronic}\addlegendentry{Drehzahl Elektronik}
          \addlegendimage{/pgfplots/refstyle=plot:tempo_measured}\addlegendentry{Drehzahl Messgerät}

          \addplot[line_black] table[x= time,y=ratio_el_me,col sep=comma,mark=none]{assets/messurement_results/tempo.csv}; 
          \label{plot:tempo_ratio}     
          \addlegendentry{Verhältnis}
        \end{axis}
      \end{tikzpicture}
    \end{center}
    \vspace{-3ex}
    
    \caption{Resultat der Messung}
            
    \label{fig:tempo_measurement}
  \end{figure}

  Als Vorgabe wurde eine Toleranz von $\pm$ 5\% der Drehzahl geplant. Diese Vorgabe scheint die Elektronik einzuhalten. Im rechten Bereich (ab Sekunde 15) ist das Verhältnis der beiden Messresultaten sehr nahe, wobei im linken Bereich (ab Sekunde 0 bis 15) die Eigenschaft des Messgerätes das Verhältnis stark verfälscht. Das Messgerät misst die Anzahl Pulse während einer Sekunde und berechnet daraus die Drehzahl. Die Elektronik misst die Zeit zwischen zwei Pulsen und berechnet daraus die Drehzahl.
