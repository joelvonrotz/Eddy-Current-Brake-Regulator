\subsection{Drehzahlmessung}\label{cap:diskussion_drehzahlmessung}
Die Ergebnisse der Messung sind zufriedenstellend. Es war möglich, die Drehzahl genau messen zu können. Die Grundstruktur des Drehzahlmessers ist in einem guten Zustand und kann als Grundbaustein für ein zukünftiges Model der Wirbelstrombremse dienen. Die Messung übertrafen die Erwartungen und konnte in den anderen Messungen erfolgreich eingesetzt werden.
\newpara
Der einzige Nachteil, welcher nicht in den Resultaten erkannt wurde, waren schwankende Messwerte. Während der Programmierung hatte der Drehzahlmesser Probleme mit den Reaktionen. Der gemessene Wert schwankte oft im Bereich von ±2.5ms von der realen Drehzahl. Dies konnte während der Entwicklung auf ca. ±0.2ms reduziert werden.
\newpara
Als Version 1 der Bremse reicht die Messung, aber für Version 2 oder 3 wird sich die Genauigkeit verbessern müssen und die Schwankungen müssten auch möglichst gelöst werden. 