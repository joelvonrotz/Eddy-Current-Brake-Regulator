\subsection{Magnetbremsung}\label{cap:diskussion_magnetbremsung}
Zu Beginn waren die Resultate nicht zufriedenstellend. Es drehte zu langsam, nicht nach den Projektvorgaben. Die Resultate galten daher als unbrauchbar. Nachdem erkannt wurde, dass die langsame Geschwindigkeit des Rades durch die Trägheit der Bremsscheibe und des Fitnessrades, an welchem die Bremsscheibe befestigt wurde, beeinflussten, wurde das Rad angepasst. 
\newpara
Nach der Anpassung waren die Messungen zufriedenstellender, aber nicht dennoch gemäss Erwartungen. Nach den Vorstellungen hätte die Bremse einer in praktischen Bereichen eingesetzten Wirbelstrombremse ähneln müssen. Die Vorgaben waren zu unterdimensioniert und die Wirbelstrombremse reagierte daher zu langsam. Die Resultate galten wiederum als unzufriedenstellend.
\newpara
Ein gutes Zeichen aber war, dass der Magnet funktionierte. Nur durch Entfernen von ca. 12kg Metall konnte eine Verdreifachung der Abbremsung erkannt werden. Daraus konnten einige Erkenntnisse gezogen werden. Der Magnet ist abhängig von der Last, die der Magnet bremsen muss, und durch Verkleinerung dieser Last verbessert sich die Bremskraft. Dies ist ein Meilenstein des Projekts. Man kann anhand dieser Erkenntnisse das Magnet neu dimensionieren und den Aufbau der Wirbelstrombremse neu aufbauen.