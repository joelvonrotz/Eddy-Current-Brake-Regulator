\newglossaryentry{glos:frequenz}
{
  name={Frequenz},
  description={Anzahl der Schwingungen im Verhältnis zu der dafür benötigten Zeit}
}

\newglossaryentry{glos:kinetische_energie}
{
  name={Kinetische Energie},
  description={Das ist die Energie, welche in einem sich bewegenden Körper gespeichert ist. Wird auch als Energie der Bewegung bezeichnet.}
}

\newglossaryentry{glos:induktion}
{
  name={Induktion},
  description={Zeitliche Änderung des magnetischen Flusses, die durch die Bewegung eines Leiters im Magnetfeld Erzielt wird. Im Leiter entsteht eine elektrische Spannung.}
}

\newglossaryentry{glos:wirbelstrom}
{
  name={Wirbelstrom},
  description={Bewegt sich ein elektrisch leitender Körper in einem Magnetfeld, wird in dem Teil, der sich gerade durch das Magnetfeld bewegt eine Spannung induziert. Das führt zu einer Spannungsdifferenz im Körper. Diese wird mit Wirbelströmen ausgeglichen.}
}

\newglossaryentry{glos:magnetfeld}
{
  name={Magnetfeld},
  description={Raum, in dem ein Magnet Kraft ausübt.}
}

\newglossaryentry{glos:ferromagnetisch}
{
  name={Ferromagnetisch},
  description={Stoffe, die von Magneten angezogen werden.}
}

\newglossaryentry{glos:elektronen}
{
  name={Elektronen},
  description={Negativ geladenes Teilchen n der Atomhülle.}
}

\newglossaryentry{glos:magnetischer_fluss}
{
  name={Magnetischer Fluss},
  description={Die Gesamtzahl aller Magnetischen Feldlinien in einem Magneten.}
}

\newglossaryentry{glos:c_schiene}
{
  name={C-Schiene},
  description={C-förmiges Metallstück, welches für Befestigungen von Elektronik dient.}
}

\newglossaryentry{glos:etg_100}
{
  name={ETG-100},
  description={Metall mit Eiseninhalt.}
}

\newglossaryentry{glos:mcu}
{
  name={Mikrocontroller},
  description={Ein Baustein, welcher aus Halbleitern besteht. Können meistens in kleinen simplen Geräten aufgefunden werden, wie zum Beispiel in einer Fernbedienung.}
}

\newglossaryentry{glos:fpu}
{
  name={FPU},
  description={Steht für Floating Point Unit und wird für Berechnungen von Kommazahlen verwendet.}
}